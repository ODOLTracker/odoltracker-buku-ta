\begin{center}
  \Large
  \textbf{KATA PENGANTAR}
\end{center}

\addcontentsline{toc}{chapter}{KATA PENGANTAR}

\vspace{2ex}

% Ubah paragraf-paragraf berikut dengan isi dari kata pengantar

Puji dan syukur kehadirat Tuhan Yang Maha Esa, atas segala rahmat dan karunia-Nya, sehingga penulis dapat menyelesaikan penelitian ini yang berjudul \tatitle

Penelitian ini disusun dalam rangka pemenuhan Tugas Akhir sebagai syarat kelulusan Mahasiswa ITS. Oleh karena itu, penulis mengucapkan banyak terima kasih kepada Oleh karena itu, penulis mengucapkan terima kasih kepada:

\begin{enumerate}[nolistsep]
  \item Bapak \headofdepartment selaku Kepala Departemen Teknik Komputer, Fakultas Teknologi Elektro dan Informatika Cerdas, Institut Teknologi Sepuluh Nopember.
  \item Bapak \advisor selaku Dosen Pembimbing I dan Ibu \coadvisor selaku Dosen Pembimbing II yang telah memberikan arahan dan membantu penulis selama pengerjaan tugas akhir ini.
  \item Bapak-Ibu dosen pengajar Departemen Teknik Komputer, atas ilmu yang telah diberikan kepada penulis selama menjalani masa perkuliahan.
  \item Kedua Orangtua, Adik, Kakek, Nenek, dan keluarga yang telah memberikan doa serta dukungan selama penulis mengenyam pendidikan.
  \item Dinas Perhubungan Provinsi Jawa Timur yang memberikan izin pengambilan data di Jalan Tanjung Perak.
  \item Pihak PT Jasa Marga selaku pengelola Gerbang Tol Dupak 2 yang telah memberikan izin untuk implementasi sistem di Gerbang Tol Dupak 2.
  \item Rekan setopik tugas akhir, Javier Janeti Suprantiyo dan Muhammad Irsyad Rafi Saputra, yang telah menemani penulis dalam pengambilan data di Jalan Tanjung Perak.
  \item Orang terdekat yang selalu ada, memberikan dukungan moral, motivasi, dan menemani dalam suka dan duka selama proses pengerjaan tugas akhir ini.
  \item Tidak lupa kepada teman - teman Teknik Komputer ITS lain yang memberikan motivasi dan semangat selama masa perkuliahan.
\end{enumerate}

Akhir kata, semoga penelitian ini dapat memberikan manfaat kepada banyak pihak. Penulis menyadari jika tugas akhir ini masih jauh dari kata sempurna. Untuk itu penulis mengharapkan saran dan kritik yang bersifat membangun untuk dapat menuai hasi yang lebih baik lagi.

\begin{flushright}
  \begin{tabular}[b]{c}
    \place{}, \MONTH{} \the\year{} \\
    \\
    \\
    \\
    \\
    \name{}
  \end{tabular}
\end{flushright}