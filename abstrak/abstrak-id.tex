\begin{center}
  \large\textbf{ABSTRAK}
\end{center}

\addcontentsline{toc}{chapter}{ABSTRAK}

\vspace{2ex}

\begingroup
% Menghilangkan padding
\setlength{\tabcolsep}{0pt}

\noindent
\begin{tabularx}{\textwidth}{l >{\centering}m{2em} X}
  Nama Mahasiswa    & : & \name{}         \\

  Judul Tugas Akhir & : & \tatitle{}      \\

  Pembimbing        & : & 1. \advisor{}   \\
                    &   & 2. \coadvisor{} \\
\end{tabularx} 
\endgroup

% Ubah paragraf berikut dengan abstrak dari tugas akhir
Penggunaan truk sebagai moda transportasi barang di Indonesia terus meningkat, namun pelanggaran \emph{ODOL} (\emph{Overdimension Overloading}) pada truk menjadi salah satu penyebab utama kecelakaan lalu lintas. Penelitian ini bertujuan mengembangkan sistem deteksi kendaraan over\-dimension menggunakan teknologi \emph{deep learning}, khususnya \emph{Convolutional Neural Network} (CNN) dengan arsitektur SSD-MobileNetV2, yang dapat berjalan secara \emph{real-time} di \emph{edge device}. Model dilatih dengan dataset kendaraan yang dianotasi, mencapai nilai mAP 0,805 dan akurasi deteksi 80,72\%. Pengujian pada dua jenis perangkat edge menunjukkan NVIDIA Jetson Nano memberikan performa terbaik dengan kecepatan inferensi 46,86 FPS, jauh lebih tinggi dibanding Beelink Gemini T34 (3,63 FPS). Sistem diimplementasikan di Gerbang Tol Dupak 2, Surabaya dengan tingkat keberhasilan transfer data mencapai 100\% meskipun terdapat keterbatasan bandwidth. Dibandingkan penelitian sejenis, sistem ini memiliki keunggulan dalam keseimbangan akurasi dan kecepatan, serta efisiensi penggunaan bandwidth (4 KB/s). Sistem juga diintegrasikan dengan backend cloud dan dilengkapi fitur notifikasi otomatis kepada pihak berwenang saat pelanggaran terdeteksi. Dengan peningkatan akurasi dan efisiensi, sistem ini memberikan solusi efektif untuk mendeteksi kendaraan \emph{ODOL} di berbagai lokasi dengan keterbatasan konektivitas.

Kata Kunci: \emph{Overdimension}, \emph{ODOL}, \emph{deep learning}, SSD-MobileNetV2, \emph{edge device}, \emph{real-time detection}, \emph{cloud integration}, Jetson Nano